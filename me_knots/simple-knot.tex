\subsection{Простой узел}

\drawknot{../schemes/s-01.txt}{a:0}{simple-knot-1.eps}
\drawknot{../schemes/s-02.txt}{a:0}{simple-knot-2.eps}

\subsection{Кровавый узел}

Похож на простой, но добавлен ещё один проход ходового кольца через петлю.

\drawknot{../schemes/s-01.txt}{a:1}{blood-knot-1.eps}

При затяжке свободная часть петли обматывается вокруг внутреннего плетения, получается форма бочонка.

\drawknot{../schemes/s-02.txt}{a:1}{blood-knot-2.eps}

Можно добавить больше проходов

\drawknot{../schemes/s-01.txt}{a:2}{blood-knot-sup-1.eps}

но при этом увеличивается только длина узла, а ширина не увеличивается.

\subsection{Восьмёрка}

\drawknot{../schemes/eight.txt}{a:1}{eight-1.eps}
