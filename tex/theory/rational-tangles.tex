\section{Рациональные путаницы}

Рациональные путаницы были введены Джоном Конвеем в качестве основного строительного блока для построения узлов
Путаница аналогична зацеплению, за исключением того, что у неё есть свободные концы. Для удобства обозначений, будем называть свободные концы северо-западным, северо-восточным, юго-восточным и юго-западным.


Горизонтальная (вертикальная) целочисленная путаница $t_a$ ($t'_a$) - это скрутка двух горизонтальных (вертикальных) прядей $|a|$ раз в положительном или отрицательном направлении сообразно знаку $a$.

Горизонтальная сумма ($+$) двух путаниц, это путаница, полученная соединением западных концов одной путаницы с восточными концами другой.

Вертиальная сумма ($+'$) двух путаниц, это путаница, полученная соединением северных концов одной путаницы с южными коныами другой.

Обе суммы обладают переместительным свойством.

$$t_a + t_b = t_{a+b};$$
$$t'_a +' t'_b =t'_{a+b}.$$

Горизонтальная и вертикальная целочисленная путаница - это рациональная путаница.
Вертикальная или горизонтальная сумма рациональных путаниц - это рациональная путаница.

Операция $t^*_a = t_{-a}$ - изменение знака всех перекрёстков в путанице, меняет её знак.

Поворот путаницы на $90°$ в любую сторону 
$$t'_a = t_{-\frac{1}{a}}.$$

$$t''_a = t_a.$$

Таким образом, любую рациональную путаницу можно представить рациональным числом или бесконечностью.

Рациональные путаницы изотопны тогда и только тогда, когда равны их числа (Теорема Конвея).

$$t_a +' t_b = (t'_a + t'_b)'.$$