\section{Общие сведения о теории узлов}

Теория узлов относится к топологии.

Узлом в топологии называется вложение окружности в трёхметрое евклидово пространство.

Под окружностью понимаем гладкое отображение
$$f:\: \left[0;1\right]\mapsto X,\, f(0) = x_0 = f(1).$$
При таком определении, для изучения узлов удобно применять фундаментальную группу.

В отличие от верёвочного узла, у математического узла нет концов: он "завязан" на замкнутой "верёвке".

Заузливание не является свойством самой окружности, а определяется способом её вложения в трёхмерное пространство.

Два узла топологически эквивалентны, если один из них можно продеформировать в другой, причём в процессе деформации не должно возникать самопересечений.

Узел, содержащий одно или несколько самопересечений окружности, называется сингулярным узлом. Такому не соответствует никакой верёвочный узел.

Полигональным узлом называется замкнутая, связная, не имеющая самопересечений ломаная, составленная из конечного числа прямолинейных отрезков.

Узел называется ручным, если существует гомеоморфизм, отображающий его на некоторый полигональный узел. Если такого гомеоморфизма не существует, то узел называется диким. Дикий узел содержит бесконечное количество бесконечно маленьких элементов, поэтому с практической точки зрения не интересен.

Зацеплением называется вложение нескольких окружностей в трёхмерное евклидово пространство. Зацепления так же бывают ручными, дикими, полигональными и т.п. Эквивалентность зацеплений определяется аналогично.

\subsection{Проекции узлов}

Точку проекции, в которую проектируются несколько точек узла, называют кратной. Различают двойные, тройные и т.д. точки проекции, в соответствии с тем, сколько точек узла проектируется в точку проекции.

Регулярное положение - такое положение узла относительно плоскости проектирования, что
а) что никакие три точки узла не проектируются в одну точку плоскости проекции, б) имеется конечное число двойных точек проекции, в) никакая двойная точка узла не является точкой касания проекции.

В случае полигонального узла, в) никакая двойная точка не является проекцией вершины ломаной.

Таким образом, на проекции узла в регулярном положении все двойные точки являются точками самопересечения проекции узла.

Для всякого ручного узла существует эквивалентный узел, который находится в регулярном положении.
Для каждого полигонального узла, не находящегося в регулярном положении, эквивалентный узел, находящийся в регулярном положении, может быть получен сколь угодно малым вращением исходного узла. [Кроуэл, Фокс 1936, перевод Виноградова 1967]

Такое положение линии, при котором в двойной точке проекции она оказывается ближе к плоскости проектирования, чем другая линия, называется переходом, а такое, при котором она оказывается дальше - проходом. Т.е. в двойной точке проекции существует проход одной линии и переход другой. Если смотреть со стороны плоскости проекции, то одна линия переходит над другой, а другая, соответственно, проходит под первой.


