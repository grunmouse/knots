\section{Хордовая диаграмма}

Если плоскую диаграмму узла обходить из некоторой точки в некотором направлении, и нумеровать точки пересечения в порядке прохождения, а потом нарисовать окружность, нанести на неё в том же порядке пронумерованные точки, а потом соединить хордами точки, относящиеся к одному и тому же пересечению - полученная схема будет называться хордовой диаграммой.