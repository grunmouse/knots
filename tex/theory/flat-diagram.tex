\section{Плоская диаграмма}

Проекция узла в регулярном положении с указанием проходов и переходов называется плоской диаграммой узла.
Проекция узла в регулярном положении без указания проходов и переходов называется тенью узла. 

Плоская диаграмма и тень узла являются четырёхвалентными графами.

Над плоской диаграммой определены движения Рейдемейстера, которые являются эквивалентными преобразованиями.

Если спроектировать узел не на плоскость, а на сферу (или замкнуть проекцию в сферу), то получится сферическая диаграмма.

\paragraph{Теорема (Рейдемейстер)} Две плоские диаграммы $D_1$ и $D_2$ порождают эквивалентные узлы тогда и только тогда, когда $D_1$ может быть преобразована в $D_2$ конечным числом движений Рейдемейстера.



\graphicspath{{\currentpath}}

$\Omega_1$ - закручивание и раскручивание калышки

\begin{tabular}{
>{\centering\arraybackslash}m{3cm}>{\centering\arraybackslash}m{0.4cm}
>{\centering\arraybackslash}m{3cm}>{\centering\arraybackslash}m{0.4cm}
>{\centering\arraybackslash}m{3cm}
}
\includesvg{images/close-loop-l}
&
=
&
\includesvg{images/line}
&
=
&
\includesvg{images/close-loop-r}
\end{tabular}

$\Omega_2$ - занос одной линии петлёй над другой линией

\begin{tabular}{
>{\centering\arraybackslash}m{3cm}>{\centering\arraybackslash}m{0.4cm}
>{\centering\arraybackslash}m{3cm}>{\centering\arraybackslash}m{0.4cm}
>{\centering\arraybackslash}m{3cm}
}
\includesvg{images/two-loops-up}
&
=
&
\includesvg{images/two-line}
&
=
&
\includesvg{images/two-loops-down}
\end{tabular}


$\Omega_3$ - если одна линия переходит над двумя другими, и они пересекаются недалеко от этой линии (другие линии не мешают картине), то первую линию можно расположить по другую сторону от пересечения других двух.

\begin{tabular}{
>{\centering\arraybackslash}m{3cm}>{\centering\arraybackslash}m{0.4cm}
>{\centering\arraybackslash}m{3cm}
}
\includesvg{images/over-cross-top}
&
=
&
\includesvg{images/over-cross-bottom}

\end{tabular}

Альтернированный узел - это такой узел, на плоской диаграмме которого, при выбранном направлении обхода, будут происходить поочерёдно проходы и переходы.
Альтернированный узел, граф которого не имеет разбивающих вершин - нетривиален.

