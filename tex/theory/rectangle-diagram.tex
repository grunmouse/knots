\section{Прямоугольная диаграмма}
Прямоугольная диаграмма - это плоская диаграмма, преобразованная простыми деформациями к такому виду, что все линии диаграммы идут по вертикали или по горизонтали, причём вертикальные линии проходят над горизонтальными.
Кроме того, есть соглашение, что линии диаграммы не должны принадлежать одной прямой.

Над прямоугольной диаграммой можно совершить следующие преобразования:
\begin{itemize}
\item Рокировка - перемена мест параллельных линий, если они при движении не зацепляются.
\item Циклический перенос - перенос крайней линии через всю диаграмму на противоположный край.
\item Уничтожение линии - исключение несущественной ступеньки.
\end{itemize}

По теореме, доказанной Дынниковым, любой тривиальный узел полумонотонно упрощается при помощи перечисленных преобразований.