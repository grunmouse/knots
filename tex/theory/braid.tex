\section{Косы и замыкание кос}

Косу можно представить как группу верёвок, которые при движении сверху вниз сплетаются друг с другом, но нигде не поворачивают вверх.

Эквивалентные преобразования косы: движение нитей вправо или влево без их разрыва или отцепления их концов.

Замыкание косы - соединение верхних концов косы с нижними тех же номеров. Замыкание косы является узлом или зацеплением.
Любой узел можно представить в виде замыкания косы [Александер,1923].

Узел-обмотка, это ориентированный узел, для которого существует точка, которую все части плоской диаграммы узла обходят в одном направлении.

Узел-обмотка может быть представлен как замыкание косы, если за начало и конец косы принять пересечение плоской диаграммы лучём, исходящим из его центральной точки и не проходящим через двойные точки.

\subsection{Группа кос}

Косы из одинакового количества нитей образуют группу относительно операции последовательного соединения (произведения).

Единичной косой является тривиальная коса, которая не содержит переплетений. Для каждой косы существует обратная коса, такая, что их соединение даёт тривиальную косу.

Простой косой является коса, в которой переплетена только одна пара нитей. Обратная коса для простой косы - это перплетение тех же нитей