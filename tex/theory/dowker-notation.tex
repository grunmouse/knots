\section{Нотация Даукера}

Нотация Даукера-Тистлтуэйта это код, представляющий собой последовательность чётных чисел.

Чтобы построить нотацию Даукера-Тистлтуэйта нужно обойти узел из произвольной начальной точки в выбранном направлении, нумеруя каждое пересечение в порядке прохождения (каждое пересечение получит два номера) со следующей модификацией: если очередной чётный номер достаётся переходу, записываем его со знаком минус. Таким образом у нас получается массив пар номеров - чётных с нечётными. Сортируем его в порядке возрастания нечётного члена. Тогда полученная последовательность чётных членов этих пар будет нотацией Даукера-Тистлтуэйта.

Например, для простого узла запишем: \
$1,2,3,4,5,6$ - все чётные номера образуют проходы, поэтому отрицательных нет; \
выпишем пары: $(1,4),(2,5),(3,6) \Rightarrow (1,4),(3,6),(2,5)$;
получим: $4,6,2$.\
Что и требовалось доказать - простой узел гомеоморфен трилистнику.