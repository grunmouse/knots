\section{Операции}

\subsection{Узловязочные операции}

\textbf{Узловязочными операциями} будем называть такие операции, которые можно выполнить на реальной верёвке, и которые превращают один узел в другой.

На разомкнутой верёвке любой узел можно получить из тривиального узла, выполнив некоторую последовательность узловязочных операций.

Операции будем рассматривать так, как они выглядят на плоской диаграмме.

\subsubsection{Элементарные узловязочные операции}

\textbf{Простые деформации}  --- деформации участков плоской диаграммы, не уничтожающие существующих перекрёстков и не создающие новых. Такие изменения не будем считать операциями.

\textbf{Изотопные операции}  --- операции, преобразующие узел в изотопный ему. Достаточным их набором являются три операции Рейдемайстера.

\textbf{Неизотопные операции}  --- операции, преобразующие узел в неизотопный ему. Для достаточного набора нужна только одна такая операция: операция заведения конца.

\textbf{Заведением конца}  будем называть операцию, которая заключается в таком перемещении конца верёвки, чтобы на плоской диаграмме образовалось одно новое пересечение, состдствующее с этим концом, если результат не совпадает ни с одним из движений Рейдемайстера.

\subsubsection{Сложные узловязочные операции}

\textbf{Сложными} будем называть такие узловязочные операции, которые можно разложить в последовательности элементарных операций.

\textbf{Пробитие колышки} --- это две операции заведения конца: конец проходит под нижней половиной колышки, и переходит через верхнюю.

\textbf{Вязание петлёй} --- на одной из дуг формируется узкая петля, которой затем делается несколько движений $\Omega_2$.

Романенко Е.А. в своей книге "ЛУЧШИЕ УЗЛЫ. ПОДБОРКА ОТ АНАЛИТИКА" предложил следующие операции:
\begin{itemize}
\item СПРЯМЛЕНИЕ – процедура, превращающая Травяной узел в Прямой. 
\item ИНВЕРТИРОВАНИЕ (ходовых) или ИНВЕРСИЯ – узловязочная операция, при которой ходовые концы на крайнем их пересечении меняются местами. Пример: взаимо-инвертируемы узлы Травяной и Тёщин 
\item СВИВКА - Из «хантера» при помощи свивки мы получаем Инфинити
\end{itemize}
А также операции обратные им.


Операции инвертирования и свивки можно рассмотреть с точки зрения преобразования целочисленной путаницы: параллельно расположенные участки ходовых концов - это целочисленная путаница со значением 0, её замена на ненулевую целочисленную путаницу с чётным значением будет \textbf{инвертированием} - если ходовые концы сонаправлены, и \textbf{свивкой} - если они противонаправленные.

\subsection{Отображающие операции}

Это операции, которые можно выполнить только мысленно, так как узлы, которые отображаются ими друг на друга, имеют разный порядок вязки. Такие операции нужны для того, чтобы указать родство узлов.

\subsubsection{Простые отображающие операции}
\begin{itemize}
\item Переполюсовка --- назначение концам компоненты узла других ролей: коренной её конец становится ходовым, а ходовой - коренным. Романенко Е.А. называл такую операцию реверсированием, но я переименовал, чтобы избежать созвучия с названиями узловязочных операций.
\item Разрезание петли --- рабочая петля перерезается и превращается в два конца.
\item Сращивание концов --- операция, обратная разрезанию петли: два конца, находящиеся по одну сторону от всех прочих элементов плоской диаграммы, сращиваются между собой.
\end{itemize}

\subsubsection{Сложные отображающие операции}
Это операции, разложимые в последовательность операций, хотябы одна из которых является отображающей, а другие могут быть отображающими или узловязочными.

\begin{itemize}
\item Укорачивающее сращивание --- определено для узлов, ходовые концы которых расположены противонаправленно друг другу. Концы проводятся в обратном направлении, до тех пор, пока не окажутся около друг друга, после чего сращиваются.
\item Разрезание с протаскиванием --- операция, обратная укорачивающему сращиванию. Компонента узла в некоторой точке разрезается, после чего образовавшиеся концы проводятся параллельно друг другу в противоположные стороны до тех пор, пока не выйдут за пределы внутренних областей плоской диаграммы.
\end{itemize}

\subsection{Абстрактные операции}

Абстрактными будем называть такие операциями, которые определены для физически невозможных или практически бессмысленных форм узлов. Они нужны для того, чтобы иметь возможность сопоставить реальный узел с абстрактной моделью узла, пригодной для анализа методами математической теории узлов. Важным свойством абстрактной модели является то, что она заузлена так же, как исходный реальный узел, но те деформации, которым реальный узел сопротивляется по физическим причинам, его абстракция не позволяет выполнить по причинам геометрическим.

\begin{itemize}
\item Устремление конца в бесконечность --- конец верёвки дополняется прямым полубесконечным продолжением.
\item Замыкание связной компоненты --- ходовой и коренной концы компоненты узла сращиваются друг с другом.
\end{itemize}
