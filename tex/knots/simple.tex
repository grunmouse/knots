\subsection{Простейшие узлы для утолщения троса}

\subsection{Гомологический ряд простых узлов}
В узлах этой группы конец имеет образует одну петлю, а затем пересекается сам с собой нечётное число раз.
Чётное число получить невозможно - конец не уйдёт в петлю.

Простой узел образуется, если пробить колышку ходовым концом.
\begin{itemize}
\item Простой узел - один шлаг 

\item Кровавый узел - два и более шлагов 

\end{itemize}

\subsection{Гомологический ряд восьмёркообразных узлов}

Восьмёрка образуется, если сделать колышку, но перед пробивкой обнести ходовой конец вокруг коренного
Восьмёрка получится, если сделать колышку, перекрутить её ещё на полоборота, а потом пробить.
Если накинуть оборот, получится стивидорный узел

Узлы, подобные восьмёрке содержат две группы пересечений из чётного числа каждое.

\begin{itemize}
\item Восьмёрка - один шлаг вокруг корневого конца, один проход через колышку

\item Стивидорныый узел - два шлага вокруг корневого конца, один проход через колышку

\item Симметричный стивидорный узел - два шлага вокруг корневого конца, два прохода через колышку
\end{itemize}

\subsection{Узлы похожие на восьмёрку, но с другой структурой}

Если сделать колышку и перекрутить её ещё на оборот, получится узел, похожий на стивидорный, но с нечётным числом пересечений.

Чтобы получить симметричный ему узел, нужно завязать простой узел, затем на сделать на нём колышку и пробить её ходовым концом.

Подобный узел может содержать только нечётное число пересечений