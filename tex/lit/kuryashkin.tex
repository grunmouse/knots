\section{"Техника вязания узлов. Теория и практика".- А.Н. Куряшкин.- 2014}

Книга состоит из двух частей. В первой части автор систематизирует свойсва узла.

\begin{itemize}
\item Свойства, влияющие на прочность и безопасность узла:
\begin{enumerate}
\item Величина радиуса кривизны узла;
\item Тугость и расслабленность узла;
\item Устойчивость узла к деформации (стабильность узла);
\item Устойчивость узла к развязыванию;
\item Устойчивость узла к переменным нагрузкам;
\item Способность узла держать на верёвке любой толщины;
\item Способность узла держать на верёвке из любого материала.
\end{enumerate}
\item Свойства, не влияющие на прочность и безопасность узла:
\begin{enumerate}
\item Направление вязки узла (зеркальность) - так автор назвал хиральность;
\item Симметричность (декоративность) узла;
\item Компактность и громоздкость узла;
\item Простота и сложность конструкции узла;
\item Быстрота завязывания узла;
\item Быстрота и лёгкость развязывания узла;
\item Завязывание узла под нагрузкой;
\item Использование ходового конца при завязывании петли - т.е. негомеоморфность петли тривиальному узлу;
\item Завязывание узла на коротком конце;
\item Расход верёвки на завязывание узла (верёвкоёмкость узла)
\end{enumerate}
\end{itemize}

В теоретической части автор описывает процесс и фазы развязывания узла
\begin{itemize}
\item 1-я фаза - деформация или выворачивание узла;
\item 2-я фаза - скольжение узла;
\item 3-я фаза - необратимое развязывание узла.
\end{itemize}

Указывает способы увеличения безопасности узла:
\begin{enumerate}
\item Выравнивание и аккуратное затягивание узла после завязывания;
\item Завязывание узла с клевантом;
\item Наложение большего числа шлагов;
\item Завязывание двойного узла;
\item Завязывание узла двойной верёвкой;
\item Закрепление ходового конца на узле;
\item Завязывание контрольного узла.
\end{enumerate}
Перечисленные методы не бесспорны и не универсальны, но, очевидно, имеют область применения.


В практической части содержится коллекция узлов с методами их вязки.
Автор даёт следующую классификацию узлов:
\begin{enumerate}
  \item Узлы для утолщение троса (стопорные узлы)
  \item Узлы для укорачивания троса - внутри автор выделяет цепочки
  \item Петли:
  \begin{enumerate}
    \item Петли незатягивающиеся (фиксированные петли) - внутри автор выделяет беседочные узлы
    \item Петли затягивающиеся (скользящие, бегущие узлы)
    \begin{enumerate}
      \item Затягивающиеся петли из незатягивающихся петель (арканы)
      \item Затягивающиеся петли из стопорного узла
    \end{enumerate}
    \item Петли регулируемые
    \begin{enumerate}
      \item Схватывающие регулируюмые петли
      \item Фиксирующие регулируемые петли
    \end{enumerate}
    \item Петли двойные и тройные
    \begin{enumerate}
      \item Двойные (тройные) незатягивающиеся петли
      \item перекидные петли
      \item Двойные (тройные) затягивающиеся петли
    \end{enumerate}
  \end{enumerate}
  \item Узлы для крепления троса к опоре
  \begin{enumerate}
    \item Штыки
    \item Привязывающие быстроразвязывающиеся узлы
    \item Прижимные узлы.
    \item Схватывающие узлы.
  \end{enumerate}
  \item Узлы для связывания двух тросов.
  \begin{enumerate}
    \item Узлы, связанные «одним концом»
    \item Узлы, связанные «встречным пропуском»
    \item Узлы, связанные двумя одинаковыми узлами
    \item Узлы для связывания двух тросов, в основе которых лежат два простых узла
  \end{enumerate}
\end{enumerate}