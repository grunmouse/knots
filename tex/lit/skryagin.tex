\section{"Морские узлы".- Л.Н. Скрягин.- 1982}

Книга представляет собой выдержку из коллекции узлов, собранной Л.Н. Скрягиным. Вместе с узлами даются способы их вязки и описание некоторых свойств узла.

\begin{quote} В английском языке термин “узел” издавна обозначался, в отличие от русского языка, тремя разными существительными: «knot», «bend» и «hitch». Первое обозначает переплетение или связывание ходового конца с коренным, а также и утолщение на конце троса; второе – переплетение ходовых концов двух разных тросов для связывания их в один; третье – прикрепление ходового конца троса к какому-либо предмету, например к мачте, рею, скобе, рыму или к другому тросу.
\end{quote}

\begin{quote}
Перевод названий некоторых узлов с английского на русский язык нередко вызывает затруднения, и иногда они становятся невыразительными, длинными и труднозапоминающимися. Нескольким узлам, названия которых на английском языке не выражают какого-либо конкретного смысла в их характеристике, автор придумал и привел в книге свои названия, такие, как “дубовый”, “водяной”, “тещин”, “кинжальный”, “змеиный”, “лиановый”, “щучий”, “олимпийский”, “акулий”, “лососевый”, “тунцовый” и “роликовый”. На английском языке эти узлы имеют описательные наименования, например “узел для привязывания рыболовного крючка для ловли лосося” (поэтому в книге он получил название “лососевый узел”) и т. п.
\end{quote}

Автор даёт следующую классификацию узлов:

\begin{enumerate}
\item Узлы для утолщения троса.
\item Незатягивающиеся узлы - к этой группе автор отнёс узлы для привязывания троса к опоре, которые не затягиваются под нагрузкой.
\item Узлы для связывания двух тросов.
\item Затягивающиеся узлы - к этой группе автор отнёс узлы для привязывания троса к опоре, которые под нагрузкой затягиваются.
\item Незатягивающиеся петли - петлями автор называет такие узлы, рабочая петля которых не распадается после снятия с опоры.
\item Затягивающиеся петли.
\item Быстроразвязывающиеся узлы - к этой группе автор отнёс узлы, сформированные петлёй, которую можно вытянуть за один из концов, в том числе быстроразвязывающиеся варианты узлов, описанных в других группах.
\item Особые морские узлы - особыми автор назвал узлы, имеющие узкую область применения: узлы для привязывания троса к различным предметам, узлы для укорачивания троса и петли, преобразущиеся из затягивающихся в незатягивающися (короче здесь "линеевские черви").
\item Узлы для рыболовных снастей - к этой группе относятся узлы для привязывания лески к крючку, для связывания лесок и привязки к ним поводков; некоторые из этих узлов  повторяют описанные в других группах, другие - специфичны для лесок и крючков.
\item Декоративные узлы.
\end{enumerate}